% <--- percent sign starts a comment line in LaTeX

%----------------------------------------------------------
% This is a sample assignment .tex file. Put your name,
% assignment number and the due date below, as shown.
% Before you typeset your own assignment try to preview 
% and print this one as follows:
%   1. Save this in a file, say hw.tex
%   2. do "latex hw"
%	3. latex will produce a pdf file that should be named hw.pdf.
%	4. Use any pdf viewer to view the document 
% ------------------------------------------------------------

\documentclass[11pt]{article}

\usepackage{fullpage,graphicx,latexsym,picinpar,amsbsy,amsmath,amsfonts}



\setlength{\evensidemargin}{0.1in}
\setlength{\oddsidemargin}{0.1in}
\setlength{\textwidth}{6.6in}
\setlength{\topmargin}{0.0in}
\setlength{\textheight}{8.7in}
\setlength{\headheight}{0in}
\setlength{\headsep}{0in}
\setlength{\topsep}{0in}
\setlength{\itemsep}{0in}
\renewcommand{\baselinestretch}{1.1}
\parskip=0.080in

\newcommand{\parend}[1]{{\left( #1  \right) }}
\newcommand{\spparend}[1]{{\left(\, #1  \,\right) }}
\newcommand{\angled}[1]{{\left\langle #1  \right\rangle }}
\newcommand{\brackd}[1]{{\left[ #1  \right] }}
\newcommand{\spbrackd}[1]{{\left[\, #1  \,\right] }}
\newcommand{\braced}[1]{{\left\{ #1  \right\} }}
\newcommand{\leftbraced}[1]{{\left\{ #1  \right. }}
\newcommand{\floor}[1]{{\left\lfloor #1\right\rfloor}}
\newcommand{\ceiling}[1]{{\left\lceil #1\right\rceil}}
\newcommand{\barred}[1]{{\left|#1\right|}}
\newcommand{\doublebarred}[1]{{\left|\left|#1\right|\right|}}
\newcommand{\spaced}[1]{{\, #1\, }}
\newcommand{\suchthat}{{\spaced{|}}}
\newcommand{\numof}{{\sharp}}

\newcommand{\half}{{\textstyle\frac{1}{2}}}
\newcommand{\elevenhalves}{{\textstyle\frac{11}{2}}}
\newcommand{\onethird}{{\textstyle\frac{1}{3}}}
\newcommand{\sixteenthirds}{{\textstyle\frac{16}{3}}}
\newcommand{\twentytwothirds}{{\textstyle\frac{22}{3}}}
\newcommand{\onefifth}{{\textstyle\frac{1}{5}}}
\newcommand{\threefifths}{{\textstyle\frac{3}{5}}}
\newcommand{\sixfifths}{{\textstyle\frac{6}{5}}}
\newcommand{\eightfifths}{{\textstyle\frac{8}{5}}}
\newcommand{\sixteenfifths}{{\textstyle\frac{16}{5}}}
\newcommand{\eightteenfifths}{{\textstyle\frac{18}{5}}}
\newcommand{\threetenths}{{\textstyle\frac{3}{10}}}
\newcommand{\twentysixfifteenths}{{\textstyle\frac{26}{15}}}
\newcommand{\fisefiftieths}{{\textstyle\frac{57}{50}}}
\newcommand{\ftwotfifths}{{\textstyle\frac{42}{25}}}
\newcommand{\fotwontwfifths}{{\textstyle\frac{42}{125}}}
\newcommand{\eithontwfifths}{{\textstyle\frac{83}{125}}}

\newcommand{\veps}{{\varepsilon}}
\newcommand{\Sigmastar}{{\Sigma^\ast}}

\newcounter{exnum}[section]
\newenvironment{problem}{{\vskip 0.1in
   \noindent \bf Problem\addtocounter{exnum}{1}~\arabic{exnum}.}}{\vskip 0.1in}

\newtheorem{theorem}{Theorem}
\newtheorem{definition}{Definition}
\newtheorem{corollary}{Corollary}
\newtheorem{lemma}{Lemma}
\newtheorem{fact}{Fact}
\newtheorem{claim}{Claim}

\newenvironment{proof}{{\it Proof:\/}}{$\Box$\vskip 0.1in}

\newcommand{\emparagraph}[1]{{\smallskip\noindent{\em #1\/}}}

\newcommand{\assign}{{\,\gets\,}}

%%%%%%%%%%%%%%%%%%%%%%%%%%%%%%%%%%%%%%%%%%%%%%%%%%%%%%%%%%%%%%%%%%%%%%%%%%%%%%%%%%%
%%%%%%%%%%%  LETTERS 
%%%%%%%%%%%%%%%%%%%%%%%%%%%%%%%%%%%%%%%%%%%%%%%%%%%%%%%%%%%%%%%%%%%%%%%%%%%%%%%%%%%

\newcommand{\barx}{{\bar x}}
\newcommand{\bary}{{\bar y}}
\newcommand{\barz}{{\bar z}}
\newcommand{\bart}{{\bar t}}

\newcommand{\bfP}{{\bf{P}}}

%%%%%%%%%%%%%%%%%%%%%%%%%%%%%%%%%%%%%%%%%%%%%%%%%%%%%%%%%%%%%%%%%%%%%%%%%%%%%%%%%%%
%%%%%%%%%%%%%%%%%%%%%%%%%%%%%%%%%%%%%%%%%%%%%%%%%%%%%%%%%%%%%%%%%%%%%%%%%%%%%%%%%%%
                                                                                  
\newcommand{\threehalfs}{\mbox{$\frac{3}{2}$}}   
\newcommand{\domino}[2]{\left[\frac{#1}{#2}\right]}  

%%%%%%%%%%%% complexity classes

\newcommand{\PP}{\mathbb{P}}
\newcommand{\NP}{\mathbb{NP}}
\newcommand{\PSPACE}{\mathbb{PSPACE}}
\newcommand{\coNP}{\textrm{co}\mathbb{NP}}
\newcommand{\DLOG}{\mathbb{L}}
\newcommand{\NLOG}{\mathbb{NL}}
\newcommand{\NL}{\mathbb{NL}}

%%%%%%%%%%% decision problems

\newcommand{\PCP}{\sc{PCP}}
\newcommand{\Path}{\sc{Path}}
\newcommand{\GenGeo}{\sc{Generalized Geography}}

\newcommand{\malytm}{{\mbox{\tiny TM}}}
\newcommand{\malycfg}{{\mbox{\tiny CFG}}}
\newcommand{\Atm}{\mbox{\rm A}_\malytm}
\newcommand{\complAtm}{{\overline{\mbox{\rm A}}}_\malytm}
\newcommand{\AllCFG}{{\mbox{\sc All}}_\malycfg}
\newcommand{\complAllCFG}{{\overline{\mbox{\sc All}}}_\malycfg}
\newcommand{\complL}{{\bar L}}
\newcommand{\TQBF}{\mbox{\sc TQBF}}
\newcommand{\SAT}{\mbox{\sc SAT}}

%%%%%%%%%%%%%%%%%%%%%%%%%%%%%%%%%%%%%%%%%%%%%%%%%%%%%%%%%%%%%%%%%%%%%%%%%%%%%%%%%%%
%%%%%%%%%%%%%%% for homeworks
%%%%%%%%%%%%%%%%%%%%%%%%%%%%%%%%%%%%%%%%%%%%%%%%%%%%%%%%%%%%%%%%%%%%%%%%%%%%%%%%%%%

\newcommand{\student}[2]{%
{\noindent\Large{ \emph{#1} SID {#2} } \hfill} \vskip 0.1in}

\newcommand{\assignment}[1]{\medskip\centerline{\large\bf CS 111 ASSIGNMENT {#1}}}

\newcommand{\duedate}[1]{{\centerline{due {#1}\medskip}}}     

\newcounter{problemnumber}                                                                                 


\newcounter{solutionnumber}

\newenvironment{solution}{{\vskip 0.1in \noindent
             \bf Solution~\addtocounter{solutionnumber}{1}\arabic{solutionnumber}:}}
				{\ \newline\smallskip\lineacross\smallskip}

\newcommand{\lineacross}{\noindent\mbox{}\hrulefill\mbox{}}

\newcommand{\decproblem}[3]{%
\medskip
\noindent
\begin{list}{\hfill}{\setlength{\labelsep}{0in}
                       \setlength{\topsep}{0in}
                       \setlength{\partopsep}{0in}
                       \setlength{\leftmargin}{0in}
                       \setlength{\listparindent}{0in}
                       \setlength{\labelwidth}{0.5in}
                       \setlength{\itemindent}{0in}
                       \setlength{\itemsep}{0in}
                     }
\item{{{\sc{#1}}:}}
                \begin{list}{\hfill}{\setlength{\labelsep}{0.1in}
                       \setlength{\topsep}{0in}
                       \setlength{\partopsep}{0in}
                       \setlength{\leftmargin}{0.5in}
                       \setlength{\labelwidth}{0.5in}
                       \setlength{\listparindent}{0in}
                       \setlength{\itemindent}{0in}
                       \setlength{\itemsep}{0in}
                       }
                \item{{\em Instance:\ }}{#2}
                \item{{\em Query:\ }}{#3}
                \end{list}
\end{list}
\medskip
}

%%%%%%%%%%%%%%%%%%%%%%%%%%%%%%%%%%%%%%%%%%%%%%%%%%%%%%%%%%%%%%%%%%%%%%%%%%%%%%%%%%%
%%%%%%%%%%%%% for quizzes
%%%%%%%%%%%%%%%%%%%%%%%%%%%%%%%%%%%%%%%%%%%%%%%%%%%%%%%%%%%%%%%%%%%%%%%%%%%%%%%%%%%

\newcommand{\quizheader}{ {\large NAME: \hskip 3in SID:\hfill}
                                \newline\lineacross \medskip }

%\newcommand{\namespace}{ {\large NAME: \hskip 3in SID:\hfill}
%                               \newline\lineacross \medskip }

%%%%%%%%%%%%%%%%%%%%%%%%%%%%%%%%%%%%%%%%%%%%%%%%%%%%%%%%%%%%%%%%%%%%%%%%%%%%%%%%%%%
%%%%%%%%%%%%% for final
%%%%%%%%%%%%%%%%%%%%%%%%%%%%%%%%%%%%%%%%%%%%%%%%%%%%%%%%%%%%%%%%%%%%%%%%%%%%%%%%%%%

\newcommand{\namespace}{\noindent{\Large NAME: \hfill SID:\hskip 1.5in\ }\\\medskip\noindent\mbox{}\hrulefill\mbox{}}





\begin{document}
	
% v -- YOUR NAME and SID in the braces
\student{Nathan Ha}{862377326}  
% v -- YOUR NAME and SID in the braces
% v -- ASSIGNMENT NUMBER in the braces
\assignment{ 1 } 
% v-- DUE DATE in the braces
\duedate{January 22, 2024 }  

\medskip

%%%%%%%%%%%%%%%%%%%%%%%%%%%%%%%%%%%%%%%%%%%%%%%%%%%%%%%%%%%%%%%%%%%%%%%%%%

\lineacross

%%%%%%%%%%%%%%%%%%%%%%%%%%%%
 % problem 1
\begin{problem}
	Give an asymptotic estimate  for the number $h(n)$ of ``Hello''s printed by Algorithm~\textsc{PrintHellos} below.
Your solution \emph{must} consist of the following steps:
%
\begin{description}
\item{(a)} First express $h(n)$ using the summation notation $\sum$.
\item{(b)} Next, give a closed-form expression
    for $h(n)$. 
\item{(c)}  Finally, give the asymptotic value of $h(n)$ using the $\Theta$-notation.
\end{description}
\noindent
Show your work and include justification for each step. 



\begin{tabbing}
aa \= aa \= aa \= aa \= aa \= aa \= \kill
\textbf{Algorithm} \textsc{PrintHellos} $(n: \mbox{\bf integer})$ \\
      \> \textbf{for} $i \leftarrow 1$ \textbf{to} $n$ \textbf{do} \\
      \> \> \textbf{for} $j \leftarrow 1$ \textbf{to} $3i^2+i$ \textbf{do} print(``Hello") \\
      \> \textbf{for} $i \leftarrow 1$ \textbf{to} $2n^2$
                         \textbf{do} \\
      \> \> \textbf{for} $j \leftarrow 1$ \textbf{to} $i$ \textbf{do}  print(``Hello") 
\end{tabbing}

\noindent
\emph{Note:} If you need any summation formulas for this problem, you are allowed to look them up. You do not need to
prove them, you can just state  in the assignment when you use them.
\end{problem}

%---------------------------

% solution 1
\begin{solution}

\begin{description}
\item{(a)} In summation notation, the number of "Hello's" that will be printed, $h(n)$, is expressed as $\mathbf{h(n)=\sum_{i=1}^{n}(\sum_{j=1}^{3i^2+i}1) + \sum_{i=1}^{2n^2}(\sum_{j=1}^{i}1)}$. The 1's inside of the innermost sums represent one "Hello" being printed. If you need the sum to be simplified, look at part (b).

\item{(b)} We can rewrite the sum as $h(n)=\sum_{i=1}^{n}(3i^2+i) + \sum_{i=1}^{2n^2}(i)$. We can further break down the sums: \\ $h(n)=3\sum_{i=1}^{n}(i^2) + \sum_{i=1}^{n}(i) + \sum_{i=1}^{2n^2}(i)$. 
\\ Now we can use the sum formulas that I looked up: 
\\ \text{       }(1) $\sum_{i=1}^{n}i = \frac{1}{2}n(n+1)$
\\ \text{       }(2) $\sum_{i=1}^{n}i^2 = \frac{1}{6}n(n+1)(2n+1)$
\\ By substituting each sum with these formulas, we get:
\\ $h(n) = 3(\frac{1}{6}n(n+1)(2n+1)) + \frac{1}{2}n(n+1) + \frac{1}{2}(2n^2)(2n^2+1) \\=
\frac{1}{2}(2n^3+3n^2+n) + \frac{1}{2}(n^2+n)+\frac{1}{2}(4n^4+2n^2)
= n^3 + \frac{3}{2}n^2 + \frac{1}{2}n + \frac{1}{2}n^2 + \frac{1}{2}n + 2n^4 + n^2
\\ = \mathbf{2n^4 + n^3 + 3n^2 + n}$.

\item{(c)}
To find $\Theta$, we have to find $O$ and $\Omega$: 
\\ O: $h(n) = 2n^4 + n^3 + 3n^2 + n \leq 2n^4 + n^4 + 3n^4 + n^4 = 7n^4 \implies h(n) = O(n^4)$.
\\ $\Omega$: $h(n) = 2n^4 + n^3 + 3n^2 + n \geq 2n^4 \implies h(n) = \Omega (n^4)$. 
\\ Since $h(n) = \Omega(n^4)$ and $h(n) = O(n^4)$, $h(n) = \Theta(n^4)$.

% 2n^4 + n^3 + 3n^2 + n


\end{description}
\end{solution}

%%%%%%%%%%%%%%%%%%%%%%%%%%%%

\newpage

% problem 2
\begin{problem}
	
\smallskip
\noindent
(a) Use properties of quadratic functions to prove that $3x^2 \geq (x + 1)^
2$ for all real $x \ge 4$.

\smallskip
\noindent
(b) Use mathematical induction and the inequality from part (a) to prove that $3^n \ge 2^{n} + 3n^2$ for all integers $n\ge 4$.

\smallskip
\noindent
(c)
Let $g(n) =  2^{n} + 3n^2$ and $h(n) = 3^n$.
Using the inequality from part~(b), prove that $g(n) = O(h(n))$.
You need to give a rigorous proof derived directly from 
the definition of the $O$-notation, without using any theorems from class.
(First, give a complete statement of the definition. 
Next, show how $g(n) = O(h(n)) $ follows from this definition.)

\end{problem}

%-----------------------------
% solution 2
\begin{solution}
\begin{description}

\item{(a)} First, we can rearrange the equation to: $3x^2 - (x+1)^2 \ge 0$, then simplifying the left hand side, we get: $3x^2 - (x^2 + 2x + 1) = 2x^2 - 2x - 1 = f(x)$. Now we have $f(x) \ge 0$. We know from the properties of quadratic functions, that the function $f(x)$ represents an parabola that opens upwards. We also know from the properties of quadratic functions that these parabolas are monotonically increasing after a certain point. This point, the vertex, has an x value of $\frac{-b}{2a}$ for polynomials of the form $ax^2 + bx + c$. The x value vertex of this parabola, $f(x)$, can then be found to be $\frac{1}{2}$. This means that: $\forall x > \frac{1}{2}$, f(x) is strictly increasing. Since $f(4) = 23$, and $4 > \frac{1}{2}$, we can see that $f(x) > 23$, $\forall x > 4$, and therefore $f(x) > 0$. This implies that $3x^2 \geq (x + 1)^2$ is true for all $ x \ge 4$.

\item{(b)} \textbf{Claim:} $3^n \ge 2^{n} + 3n^2$ for all integers $n\ge 4$
\\ \textbf{Base Case:} Let n = 4. 
\\ $3^4 \ge 2^4 + 3*4^2$
\\ $81 \ge 16 + 48$
\\ $81 \ge 64 \implies$ The statement is true for n=4.
\\ \textbf{Induction Hypothesis:} Assume the statement holds for $n=k \ge 4$: $3^k \ge 2^{k} + 3k^2$.
\\ \textbf{Inductive Step:} To show that the statement is true for n = k+1, we will multiply the entire inequality by 3 to get: $3\cdot3^k \ge 3\cdot2^{k} + 3\cdot3k^2$, which simplifies to $3^{k+1} \ge 3\cdot2^{k} + 3\cdot3k^2$. From here, we can make the following observations for $k \ge 4$:
\\ $3^{k+1} \ge 3\cdot2^{k} + 3\cdot3k^2 \ge 2\cdot2^k + 3 \cdot 3k^2 \ge 2^{k+1} + 3k^2 \ge 3k^2 \ge (k+1)^2$. The last inequality comes from the result we showed in part (a). Since $(k+1)^2$ is less than $3k^2$, and $3^{k+1} \ge 2^{k+1} + 3k^2$, we can conclude that: $3^{k+1} \ge 2^{k+1} + 3(k+1)^2$. QED.

\item {(c)} In order to prove that $g(n) = O(h(n))$, we will start with the definition: For two functions $g(n)$ and $h(n)$, $g(n) = O(h(n))$ if there are positive constants $c$ and $n_0$ such that $g(n) \le c \cdot h(n)$ for all $n \ge n_0$. Following from the result proved in part (b), we can see that $h(n) \ge g(n)$ for all integers $n \ge 4$. We can see that the conditions for the definition is satisfied, with $c=1$ and $n_0=4$, thus $g(n) = O(h(n))$.

\end{description}
\end{solution}

%%%%%%%%%%%%%%%%%%%%%%%%%%%
\newpage

% problem 3
\begin{problem}
 	Give asymptotic estimates, using the $\Theta$-notation, for the following functions:
%
\begin{description}\setlength{\itemsep}{-0.01in}
%
\item{(a)} $3n^3 - 15n^2 + 2n + 4$
\item{(b)} $3n^2\log n +  2n^2\sqrt{n} + n^2$
\item{(c)} $n\log^3n -5 n + \dfrac{n^2}{\log n}$
\item{(d)} $7 \cdot n^5 + n^3 \log^2 n + 2^n$
\item{(e)} $\log^{9}n + n^3 4^n + n 5^n$
%
\end{description}

\end{problem}

%----------------------------

\begin{solution}

\begin{description}

\item{(a)} To find $\Theta$, we have to find $O$ and $\Omega$: 
\\ O: $f(n) = 3n^3 - 15n^2 + 2n + 4 \leq 3n^3 + 2n^3 + 4n^3 = 9n^3 \implies f(n) = O(n^3)$.
\\ $\Omega$: $f(n) = 3n^3 - 15n^2 + 2n + 4 \geq 3n^3 - 15n^3 \geq -12n^3 \implies f(n) = \Omega(n^3).$
\\ Since $f(n) = \Omega(n^3)$ and $f(n) = O(n^3)$, $\mathbf{f(n) = \Theta(n^3)}$.

\item{(b)} Note: $\log^a{n} = O(n^b)$ for $ a,b > 0$
\\ O: $f(n) = 3n^2\log{n} + 2n^2\sqrt{n} + n^2 \leq 3n^2\sqrt{n} + 2n^2\sqrt{n} + n^2\sqrt{n} = 6n^2\sqrt{n} \implies f(n) = O(n^2\sqrt{n})$.
\\ $\Omega$: $f(n) = 3n^2\log{n} + 2n^2\sqrt{n} + n^2 \geq 2n^2\sqrt{n} \implies f(n) = \Omega(n^2\sqrt{n})$
\\ Since $f(n) = \Omega(n^2\sqrt{n})$ and $f(n) = O(n^2\sqrt{n})$, $\mathbf{f(n) = \Theta(n^2\sqrt{n})}$.

\item{(c)} O: $f(n) = n\log^3{n} - 5n + \frac{n^2}{\log{n}}.$ From here, we can multiply by $\frac{\log{n}}{\log{n}}$ to get \\ $f(n) = \frac{1}{\log{n}}(n\log^4{n} - 5n\log{n} + n^2) \leq \frac{1}{\log{n}}(n\cdot n + n^2) = \frac{2n^2}{\log{n}} \implies f(n) = O(\frac{n^2}{\log{n}})$.
\\$\Omega: f(n) = n\log^3{n} - 5n + \frac{n^2}{\log{n}} \geq \frac{n^2}{\log{n}} - 5n \geq \frac{n^2}{\log{n}} - 5n\frac{n}{\log{n}} = \frac{-4n^2}{\log{n}} \implies f(n) = \Omega(\frac{n^2}{\log{n}})$.
\\ Since $f(n) = \Omega(\frac{n^2}{\log{n}})$ and $f(n) = O(\frac{n^2}{\log{n}})$, $\mathbf{f(n) = \Theta(\frac{n^2}{\log{n}})}$.

\item{(d)} Note: $n^a = O(2^n)$, for $a > 0$
\\ O: $f(n) = 7n^5 + n^3 \log^2 n + 2^n \leq 7\cdot 2^n + n^3 \cdot n + 2^n \leq 7\cdot 2^n + 2^n + 2^n = 9\cdot 2^n \implies f(n) = O(2^n)$.
\\ $\Omega: f(n) = 7n^5 + n^3 \log^2 n + 2^n \geq 2^n \implies f(n) = \Omega (2^n)$   .
\\ Since $f(n) = \Omega(2^n)$ and $f(n) = O(2^n)$, $\mathbf{f(n) = \Theta(2^n})$.

\item{(e)} Note: $n^a = O(nb^n)$, for constants $ a,b > 0$
\\ O: $f(n) = \log^{9}n + n^3 4^n + n 5^n \leq n + n^2\cdot n4^n + n5^n \leq n5^n + (\frac{5}{4})^n \cdot n4^n + n5^n = n5^n + n5^n + n5^n \\ = 3n5^n \implies f(n) = O(n5^n)$.
\\ $\Omega$: $f(n) = \log^{9}n + n^3 4^n + n 5^n \geq n5^n \implies f(n) = \Omega(n5^n)$.
\\ Since $f(n) = \Omega(n5^n)$ and $f(n) = O(n5^n)$, $\mathbf{f(n) = \Theta(n5^n})$.


\end{description}

\end{solution}

\paragraph{Academic integrity declaration.}
I did this homework individually. That being said, I got a lot of help from the TA (Ezekiel) during office hours.

\end{document}
