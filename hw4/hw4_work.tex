% <--- percent sign starts a comment line in LaTeX

%----------------------------------------------------------
% This is a sample assignment .tex file. Put your name,
% assignment number and the due date below, as shown.
% Before you typeset your own assignment try to preview 
% and print this one as follows:
%   1. Save this in a file, say hw.tex
%   2. do "latex hw"
%	3. latex will produce a pdf file that should be named hw.pdf.
%	4. Use any pdf viewer to view the document 
% ------------------------------------------------------------

\documentclass[11pt]{article}

\usepackage{fullpage,graphicx,latexsym,picinpar,amsbsy,amsmath,amsfonts}



\setlength{\evensidemargin}{0.1in}
\setlength{\oddsidemargin}{0.1in}
\setlength{\textwidth}{6.6in}
\setlength{\topmargin}{0.0in}
\setlength{\textheight}{8.7in}
\setlength{\headheight}{0in}
\setlength{\headsep}{0in}
\setlength{\topsep}{0in}
\setlength{\itemsep}{0in}
\renewcommand{\baselinestretch}{1.1}
\parskip=0.080in

\newcommand{\parend}[1]{{\left( #1  \right) }}
\newcommand{\spparend}[1]{{\left(\, #1  \,\right) }}
\newcommand{\angled}[1]{{\left\langle #1  \right\rangle }}
\newcommand{\brackd}[1]{{\left[ #1  \right] }}
\newcommand{\spbrackd}[1]{{\left[\, #1  \,\right] }}
\newcommand{\braced}[1]{{\left\{ #1  \right\} }}
\newcommand{\leftbraced}[1]{{\left\{ #1  \right. }}
\newcommand{\floor}[1]{{\left\lfloor #1\right\rfloor}}
\newcommand{\ceiling}[1]{{\left\lceil #1\right\rceil}}
\newcommand{\barred}[1]{{\left|#1\right|}}
\newcommand{\doublebarred}[1]{{\left|\left|#1\right|\right|}}
\newcommand{\spaced}[1]{{\, #1\, }}
\newcommand{\suchthat}{{\spaced{|}}}
\newcommand{\numof}{{\sharp}}

\newcommand{\half}{{\textstyle\frac{1}{2}}}
\newcommand{\elevenhalves}{{\textstyle\frac{11}{2}}}
\newcommand{\onethird}{{\textstyle\frac{1}{3}}}
\newcommand{\sixteenthirds}{{\textstyle\frac{16}{3}}}
\newcommand{\twentytwothirds}{{\textstyle\frac{22}{3}}}
\newcommand{\onefifth}{{\textstyle\frac{1}{5}}}
\newcommand{\threefifths}{{\textstyle\frac{3}{5}}}
\newcommand{\sixfifths}{{\textstyle\frac{6}{5}}}
\newcommand{\eightfifths}{{\textstyle\frac{8}{5}}}
\newcommand{\sixteenfifths}{{\textstyle\frac{16}{5}}}
\newcommand{\eightteenfifths}{{\textstyle\frac{18}{5}}}
\newcommand{\threetenths}{{\textstyle\frac{3}{10}}}
\newcommand{\twentysixfifteenths}{{\textstyle\frac{26}{15}}}
\newcommand{\fisefiftieths}{{\textstyle\frac{57}{50}}}
\newcommand{\ftwotfifths}{{\textstyle\frac{42}{25}}}
\newcommand{\fotwontwfifths}{{\textstyle\frac{42}{125}}}
\newcommand{\eithontwfifths}{{\textstyle\frac{83}{125}}}

\newcommand{\veps}{{\varepsilon}}
\newcommand{\Sigmastar}{{\Sigma^\ast}}

\newcounter{exnum}[section]
\newenvironment{problem}{{\vskip 0.1in
   \noindent \bf Problem\addtocounter{exnum}{1}~\arabic{exnum}.}}{\vskip 0.1in}

\newtheorem{theorem}{Theorem}
\newtheorem{definition}{Definition}
\newtheorem{corollary}{Corollary}
\newtheorem{lemma}{Lemma}
\newtheorem{fact}{Fact}
\newtheorem{claim}{Claim}

\newenvironment{proof}{{\it Proof:\/}}{$\Box$\vskip 0.1in}

\newcommand{\emparagraph}[1]{{\smallskip\noindent{\em #1\/}}}

\newcommand{\assign}{{\,\gets\,}}

%%%%%%%%%%%%%%%%%%%%%%%%%%%%%%%%%%%%%%%%%%%%%%%%%%%%%%%%%%%%%%%%%%%%%%%%%%%%%%%%%%%
%%%%%%%%%%%  LETTERS 
%%%%%%%%%%%%%%%%%%%%%%%%%%%%%%%%%%%%%%%%%%%%%%%%%%%%%%%%%%%%%%%%%%%%%%%%%%%%%%%%%%%

\newcommand{\barx}{{\bar x}}
\newcommand{\bary}{{\bar y}}
\newcommand{\barz}{{\bar z}}
\newcommand{\bart}{{\bar t}}

\newcommand{\bfP}{{\bf{P}}}

%%%%%%%%%%%%%%%%%%%%%%%%%%%%%%%%%%%%%%%%%%%%%%%%%%%%%%%%%%%%%%%%%%%%%%%%%%%%%%%%%%%
%%%%%%%%%%%%%%%%%%%%%%%%%%%%%%%%%%%%%%%%%%%%%%%%%%%%%%%%%%%%%%%%%%%%%%%%%%%%%%%%%%%
                                                                                  
\newcommand{\threehalfs}{\mbox{$\frac{3}{2}$}}   
\newcommand{\domino}[2]{\left[\frac{#1}{#2}\right]}  

%%%%%%%%%%%% complexity classes

\newcommand{\PP}{\mathbb{P}}
\newcommand{\NP}{\mathbb{NP}}
\newcommand{\PSPACE}{\mathbb{PSPACE}}
\newcommand{\coNP}{\textrm{co}\mathbb{NP}}
\newcommand{\DLOG}{\mathbb{L}}
\newcommand{\NLOG}{\mathbb{NL}}
\newcommand{\NL}{\mathbb{NL}}

%%%%%%%%%%% decision problems

\newcommand{\PCP}{\sc{PCP}}
\newcommand{\Path}{\sc{Path}}
\newcommand{\GenGeo}{\sc{Generalized Geography}}

\newcommand{\malytm}{{\mbox{\tiny TM}}}
\newcommand{\malycfg}{{\mbox{\tiny CFG}}}
\newcommand{\Atm}{\mbox{\rm A}_\malytm}
\newcommand{\complAtm}{{\overline{\mbox{\rm A}}}_\malytm}
\newcommand{\AllCFG}{{\mbox{\sc All}}_\malycfg}
\newcommand{\complAllCFG}{{\overline{\mbox{\sc All}}}_\malycfg}
\newcommand{\complL}{{\bar L}}
\newcommand{\TQBF}{\mbox{\sc TQBF}}
\newcommand{\SAT}{\mbox{\sc SAT}}

%%%%%%%%%%%%%%%%%%%%%%%%%%%%%%%%%%%%%%%%%%%%%%%%%%%%%%%%%%%%%%%%%%%%%%%%%%%%%%%%%%%
%%%%%%%%%%%%%%% for homeworks
%%%%%%%%%%%%%%%%%%%%%%%%%%%%%%%%%%%%%%%%%%%%%%%%%%%%%%%%%%%%%%%%%%%%%%%%%%%%%%%%%%%

\newcommand{\student}[2]{%
{\noindent\Large{ \emph{#1} SID {#2} } \hfill} \vskip 0.1in}

\newcommand{\assignment}[1]{\medskip\centerline{\large\bf CS 111 ASSIGNMENT {#1}}}

\newcommand{\duedate}[1]{{\centerline{due {#1}\medskip}}}     

\newcounter{problemnumber}                                                                                 


\newcounter{solutionnumber}

\newenvironment{solution}{{\vskip 0.1in \noindent
             \bf Solution~\addtocounter{solutionnumber}{1}\arabic{solutionnumber}:}}
				{\ \newline\smallskip\lineacross\smallskip}

\newcommand{\lineacross}{\noindent\mbox{}\hrulefill\mbox{}}

\newcommand{\decproblem}[3]{%
\medskip
\noindent
\begin{list}{\hfill}{\setlength{\labelsep}{0in}
                       \setlength{\topsep}{0in}
                       \setlength{\partopsep}{0in}
                       \setlength{\leftmargin}{0in}
                       \setlength{\listparindent}{0in}
                       \setlength{\labelwidth}{0.5in}
                       \setlength{\itemindent}{0in}
                       \setlength{\itemsep}{0in}
                     }
\item{{{\sc{#1}}:}}
                \begin{list}{\hfill}{\setlength{\labelsep}{0.1in}
                       \setlength{\topsep}{0in}
                       \setlength{\partopsep}{0in}
                       \setlength{\leftmargin}{0.5in}
                       \setlength{\labelwidth}{0.5in}
                       \setlength{\listparindent}{0in}
                       \setlength{\itemindent}{0in}
                       \setlength{\itemsep}{0in}
                       }
                \item{{\em Instance:\ }}{#2}
                \item{{\em Query:\ }}{#3}
                \end{list}
\end{list}
\medskip
}

%%%%%%%%%%%%%%%%%%%%%%%%%%%%%%%%%%%%%%%%%%%%%%%%%%%%%%%%%%%%%%%%%%%%%%%%%%%%%%%%%%%
%%%%%%%%%%%%% for quizzes
%%%%%%%%%%%%%%%%%%%%%%%%%%%%%%%%%%%%%%%%%%%%%%%%%%%%%%%%%%%%%%%%%%%%%%%%%%%%%%%%%%%

\newcommand{\quizheader}{ {\large NAME: \hskip 3in SID:\hfill}
                                \newline\lineacross \medskip }

%\newcommand{\namespace}{ {\large NAME: \hskip 3in SID:\hfill}
%                               \newline\lineacross \medskip }

%%%%%%%%%%%%%%%%%%%%%%%%%%%%%%%%%%%%%%%%%%%%%%%%%%%%%%%%%%%%%%%%%%%%%%%%%%%%%%%%%%%
%%%%%%%%%%%%% for final
%%%%%%%%%%%%%%%%%%%%%%%%%%%%%%%%%%%%%%%%%%%%%%%%%%%%%%%%%%%%%%%%%%%%%%%%%%%%%%%%%%%

\newcommand{\namespace}{\noindent{\Large NAME: \hfill SID:\hskip 1.5in\ }\\\medskip\noindent\mbox{}\hrulefill\mbox{}}





\begin{document}
	
% v -- YOUR NAME and SID in the braces
\student{ Nathan Ha }{ 862377326 }  
% v -- YOUR NAME and SID in the braces
% \student{ name 2 }{sid 2 } 
% v -- ASSIGNMENT NUMBER in the braces
\assignment{ 4 } 
% v-- DUE DATE in the braces
\duedate{February 19, 2024 }  

\medskip

%%%%%%%%%%%%%%%%%%%%%%%%%%%%%%%%%%%%%%%%%%%%%%%%%%%%%%%%%%%%%%%%%%%%%%%%%%

\lineacross

%%%%%%%%%%%%%%%%%%%%%%%%%%%%


\newcommand{\calT}{{\mathcal{T}}}

% PROBLEM 1
%%%%%%%%%%%%%%%%%%%%%%%%%%%%%
\begin{problem}
Give an asymptotic estimate, using the $\Theta$-notation, of the number of letters printed by the
algorithms given below. Give a complete justification for your answer, by providing an appropriate recurrence
equation and its solution.

\medskip
\noindent
(a) 
\hspace{0.01in}
%
\begin{minipage}[t]{2.4in}
\strut\vspace*{- 2.5 \baselineskip}\newline 
\input{pseudocode.tex}
\begin{program}
algorithm |PrintAs|$(n)$
   if $n\le 1$ then
      |print("A")|
   else
      for $j\assign 1$ to $n^3$
         do |print("A")|
      for $i\assign 1$ to $5$ do
         |PrintAs|$(\,\floor{n/2}\,)$
\end{program}
\end{minipage}
%
\hspace{0.4in}
(b) 
\hspace{0.01in}
%
\begin{minipage}[t]{2.4in}
\strut\vspace*{- 2.5 \baselineskip}\newline 
\input{pseudocode.tex}
\begin{program}
algorithm |PrintBs|$(n)$
   if $n\ge 4$ then
      for $j\assign 1$ to $n^2$
         do |print("B")|
      for $i\assign 1$ to $6$ do
         |PrintBs|$(\,\floor{n/4}\,)$
      for $i\assign 1$ to $10$ do
         |PrintBs|$(\,\ceiling{n/4}\,)$
\end{program}
\end{minipage}

\medskip
\noindent
(c) 
\hspace{0.01in}
%
\begin{minipage}[t]{2.4in}
\strut\vspace*{- 2.5 \baselineskip}\newline 
\input{pseudocode.tex}
\begin{program}
algorithm |PrintCs|$(n)$
   if $n\le 2$ then
      |print("C")|
   else
      for $j\assign 1$ to $n$
         do |print("C")|
      |PrintCs|$(\,\floor{n/3}\,)$
      |PrintCs|$(\,\floor{n/3}\,)$
      |PrintCs|$(\,\floor{n/3}\,)$
      |PrintCs|$(\,\floor{n/3}\,)$
\end{program}
\end{minipage}
%
\hspace{0.4in}
(d) 
\hspace{0.01in}
%
\begin{minipage}[t]{2.4in}
\strut\vspace*{- 2.5 \baselineskip}\newline 
\input{pseudocode.tex}
\begin{program}
algorithm |PrintDs|$(n)$  
   if $n\ge 5$ then
      |print("D")|
      |print("D")|
     if $(x \equiv 0 \pmod 2)$ then 
         |PrintDs|$(\,\floor{n/5}\,)$
         |PrintDs|$(\,\ceiling{n/5}\,)$
         $x\assign \ x + 3$
      else
         |PrintDs|$(\,\ceiling{n/5}\,)$
         |PrintDs|$(\,\floor{n/5}\,)$
         $x\assign 5x + 3$
\end{program}
\end{minipage}

\noindent
In part~(d), variable $x$ is a global variable initialized to $1$.
\end{problem}

%---------------------------
% solution 1
\text{}\\
The solutions for all the problems will use the master theorem, so I will just state it here. Let $a \geq 1, b > 1, c > 0$ and $d \geq 0$. If $T(n)$ satisfies the recurrence $T(n) = aT(n/b) + cn^d$.
% T(n) = {...
\[ T(x) = \begin{cases} 
   \Theta(n^{log_b{(a)}}) & a > b^d \\
   \Theta(n^dlog(n)) & a = b^d \\
   \Theta(n^d) & a < b^d 
\end{cases}
\]

\newpage

\begin{solution}
   \\
	(a) This algorithm can be rewritten as the recurrence: $A(n) = n^3 + 5A(\floor{n/2})$. Applying the master theorem, with $a = 5, b=2, c=1, d=3$. Since $a < b^d$, $T(n) = \Theta(n^3)$.
   \\\\
   (b) This algorithm can be rewritten as the recurrence: $B(n) = n^2 + 6B(\floor{n/4}) + 10B(\lceil{n/4}\rceil) = 16B(n/4) + n^2 \implies a=16, b=4, c=1, d=2$. From the master theorem, since $a=b^d$, $B(n) = \Theta(n^2\log(n))$.
   \\\\
   (c) This algorithm can be rewritten as the recurrence: $C(n) = 4C(\floor{n/3}) + n$. Applying the master theorem, with $a=4, b=3, c=1, d=1 \implies a > b^d \implies C(n) = \Theta(n^{\log_3{4}})$.
   \\\\
   (d) This algorithm can be rewritten as the recurrence: $D(n) = 2 + 2D(n/5)$. The reason for the $2D(n/5)$ is because there even though there are four recursive calls in the code, only two are executed in any given iteration. Applying the master theorem, with $a=2,b=5,c=2,d=0 \implies a > b^d$, we have $B(n) = \Theta(n^{log_5(2)})$.
\end{solution}

%%%%%%%%%%%%%%%%%%%%%%%%%%%%

% PROBLEM 2
\newpage
\vspace{0.1in}
\begin{problem}
We have three sets $A$, $B$, $C$ with the following properties:

\begin{description}

\item{(a)}  $|B| = 2|A|$, $|C| = 3|A|$, 


\item{(b)} $|A\cap B| = 18$,
        $|A\cap C| = 20$,
        $|B\cap C| = 24$,

\item{(c)}
$|A\cap B\cap C| = 11$,

\item{(d)}
$|A\cup B\cup C| = 129$.

\end{description}

\noindent Use the inclusion-exclusion principle to determine the number of elements in $A$. 
Show your work.
\end{problem}


%-----------------------------

\begin{solution}
	According to the inclusion exclusion principle, we have: \\ $|A \cup B \cup C| = |A| + |B| + |C| - |A \cap B| - |A \cap C| - |B \cap C| + |A \cap B \cap C|$.
   \\
   We can substitute the values from (b), (c), (d) into the principle:
   \\
   $129 = |A| + |B| + |C| - 18 - 20 - 24 + 11 \implies |A| + |B| + |C| = 180$. To solve this, we want to put everything in terms of a single variable. From (a), we know the relationships between the three variables. I will choose to express everything in terms of $|A|$. Plugging this in, we get 
   \\ 
   $|A| +  2|A| + 3|A| = 180 \implies |A| = 30$. 

\end{solution}

% PROBLEM 3
\newpage
%%%%%%%%%%%%%%%%%%%%%%%%%%%

\begin{problem}
	A company, Nice Inc., will award 45 fellowships to high-achieving UCR students from four different majors: computer science, biology, political science and history. They decided to give fellowship awards to at least 8 students majoring in computer science and at most 8 biology majors. The number of political science and history majors should be between 5 and 12 students each. How many possible lists of awardees are there? You need to give a complete derivation for the final answer, using the method developed in class. 
	(Brute force listing of all lists will not be accepted.)
	
\end{problem}
	

%----------------------------
% solution 3
\begin{solution}
   Let's say that the number of CS awards is C, biology is B, political science is P, history is H. We can express the conditions as:
   \\
   $
      C \geq 8 \\
      B \leq 8 \\
      5 \leq P, H, \leq 12 \\
      C + B + P + H = 45
   $
   \\\\
   We can substitute $B' = B - 5$ and $P' = P - 5$  $(0 \leq P', H' \leq 7)$ into the sum. Also, since C has no restrictions on the upper bound, we can kind of exclude it for now\footnote[1]{As you may notice, later on when we finding the number of integer partitions, we choose k=4, which includes C. So how could we account for C in k, yet exclude it in m? This is because we are essentially setting aside 8 slots for C (which is why we subtract 8 from 45), and then seeing how many integer partitions exist. Due to the way the partition formula works, there's no reason that $B + P' + H'$ cannot actually add up to less than 27. This is because we are assuming there is a fourth group (C) that will fill that void.} (and subtract 8). We can now write:
   \\
   $B + P' + H' = 27$.
   \\\\
   We can also let the number of partitions that fit these conditions be the total number of partitions $(S())$ minus the number of invalid partitions:
   \\
   $
      S(B \leq 8 \wedge P'\leq 7 \wedge H' \leq 7) 
      \\= S() - S(B \geq 9 \lor P' \geq 8 \lor H' \geq 8) 
   $.
   \\\\
   We know that the total number of partitions, 
   \\
   S() = ${m + k - 1 \choose k - 1}, m=27, k=4 \implies S() = {30 \choose 3} = 4060$.
   \\
   m is the total number of awards, and k is the number of groups.
   \\\\
   We can also calculate the number of invalid partitions using the inclusion-exclusion principle. Also, the number of partitions with conditions is ${m - A + k - 1 \choose k - 1}$, where A is the sum of the lower bounds.
   \\
   $
      S(B \geq 9 \lor P' \geq 8 \lor H' \geq 8) = 
      \\S(B \geq 9) + S(P' \geq 8) + S(H' \geq 8) \\
      - S(B \geq 9 \land P' \geq 8)
      - S(B \geq 9 \land P' \geq 8)
      - S(B \geq 8 \land P' \geq 8)\\
      + S(B \geq 9 \land P' \geq 8 \land H' \geq 8)
      \\\\={27 - 9 + 4 - 1 \choose 4 - 1} + {27 - 8 + 4 - 1 \choose 4 - 1} + {27 - 8 + 4 - 1 \choose 4 - 1}
      \\-{27 - (9+8) + 4 - 1 \choose 4 - 1} -{27 - (9+8) + 4 - 1 \choose 4 - 1} - {27 - (8+8) + 4 - 1 \choose 4 - 1} - {27 - (9+8+8) + 4 - 1 \choose 4 - 1} 
      \\ + {27 -(8+8+9) + 4-1 \choose 4-1}
      \\=1330 + 1540 + 1540 - 286 - 286 - 364 + 10 = 3484
   $.
   \\
   Subtracting the total with the number of invalid partitions, we get $4060 - 3484 = 576$.




\end{solution}


%%%%%%%%%%%%%%%%%%%%%%%%%%%%
\newpage
\paragraph{Academic integrity declaration.}
I did this homework by myself. I got help from office hours, though.

%%%%%%%%%%%%%%%%%%%%%%%%%%%%


\end{document}
